%-----------------------------------------------------------------------
% Arquivo template *.tex para TCC da Especialização em Data Science &
% Big Data.
%-----------------------------------------------------------------------

% Classe do documento.
\documentclass[9pt, a4paper, twocolumn]{article}

%-----------------------------------------------------------------------
% Metadados do documento. ----------------------------------------------

\title{$title$}
\author{
  $author$\footnotemark[1]\\
  %Orientador Externo se Houver\footnotemark[2]\\
  $orientador$\footnotemark[2]\\
  %Outro Orientador do Programa\footnotemark[3]
}

% Define as chamadas dos autores.
\def\articlefootnotes{
  \let\thefootnote\relax\footnotetext{$role$, \href{emailto:juniorssz@gmail.com}{$email$}.}
  %\let\thefootnote\relax\footnotetext{Chefe de Departamento de Data Science da Empresa X, \url{chefe@empresax.com}.}
  \let\thefootnote\relax\footnotetext{$orientador_role$.}
}
\date{}

% Define variáveis usadas no `dsbd_capa.tex`.
\def\capaautor{$author$}
\def\capaorientador{$orientador$}
\def\capaano{$ano$}
\makeatletter
\let\capatitulo\@title
\makeatother

%-----------------------------------------------------------------------
% Preâmbulo. -----------------------------------------------------------

% Define variáveis usadas no `dsbd_preamble.tex`.
\def\pathtologoufpr{src/logo-ufpr.png}
\def\pathtologodsbd{src/dsbd1x4.png}

% Não alterar o conteúdo do arquivo abaixo.
%-----------------------------------------------------------------------

\usepackage[top=2.5cm, left=1.75cm, right=1.75cm, bottom=2cm]{geometry}
\usepackage[brazil]{babel}
\usepackage[utf8x]{inputenc}

\usepackage{booktabs}

\usepackage{amsmath, amsfonts, amssymb, amsthm}
\renewcommand{\labelitemi}{\raisebox{0.3ex}{\footnotesize{$\blacktriangleright$}}}
\usepackage{enumitem}
\setlist{itemsep = -2pt}
\usepackage{setspace}

\usepackage{graphicx}
\usepackage[svgnames]{xcolor}
\usepackage[colorlinks, citecolor = DarkRed, linkcolor = DarkGreen, urlcolor = DarkBlue]{hyperref}

\usepackage{lmodern}
% \usepackage{palatino}
\usepackage{mathpazo}
\usepackage[T1]{fontenc}
\usepackage[scaled=0.8]{beramono}

% Padrão de fontes para títulos de sessão e similares.
\usepackage{titlesec}
\titleformat*{\section}{\Large\bfseries\sffamily}
\titleformat*{\subsection}{\large\bfseries\sffamily}
\titleformat*{\subsubsection}{\bfseries\sffamily}

\usepackage{bookmark}       % Para ter pontos marcados no texto.
\usepackage{lipsum}         % Para texto dummy.
% \lipsum, \lipsum[3-56]

\usepackage{microtype}
\usepackage{tabularx}
\usepackage{multirow}
\usepackage{float}

\usepackage{natbib}

\makeatletter
\let\@fnsymbol\@arabic
\makeatother

% Cabeçalhos e rodapés.
\usepackage{fancyhdr}
\pagestyle{fancyplain}
\fancyhf{}
\lhead{\fancyplain{}{Especialização em Data Science e Big Data $\cdot$ UFPR}}
\rhead{\fancyplain{}{\href{http://dsbd.leg.ufpr.br}{\textit{dsbd.leg.ufpr.br}}}}
\cfoot{\fancyplain{}{\thepage}}

\usepackage{titling}

\setlength{\droptitle}{-1.5cm}
\renewcommand{\maketitlehooka}{%
  \begin{minipage}[t]{4.5cm}\vspace{-0.45em}%
    \includegraphics[height = 1.5cm]{\pathtologodsbd}
  \end{minipage}
  \hspace{0.5em}
  \begin{minipage}[t]{8cm}\vspace{0pt}%
    {\large Especialização em Data Science e Big Data}\\
    {Universidade Federal do Paraná}\\
    {\href{http://dsbd.leg.ufpr.br}{\textit{dsbd.leg.ufpr.br}}}\\
  \end{minipage}
  \hfill
  \begin{minipage}[t]{2.25cm}\vspace{-0.45em}%
    \includegraphics[height = 1.5cm]{\pathtologoufpr}
  \end{minipage}

  \rule{\linewidth}{0.5pt}
  \\[2ex]
  }
\renewcommand{\maketitlehookc}{}

\pretitle{\Large\bfseries\sffamily}
\posttitle{\vskip 0.5em}

\preauthor{\begin{flushleft}}
\postauthor{\end{flushleft}}

%-----------------------------------------------------------------------

%-----------------------------------------------------------------------
% Início do documento. -------------------------------------------------

\begin{document}

%-----------------------------------------------------------------------
% Capa.

% Não alterar o conteúdo do arquivo abaixo.
\onecolumn
\thispagestyle{empty}
\begin{center}
  \linespread{1.25}
  \sffamily

  {\LARGE
    Universidade Federal do Paraná\\
    Setor de Ciências Exatas\\
    Departamento de Estatística\\
    Programa de Especialização em \emph{Data Science} e \emph{Big Data}\\
    \par
  }

  \vspace{4em}

  {\Large \capaautor}

  \vspace{21em}

  \begin{minipage}{0.9\linewidth}
    \begin{center}
      {\huge\bfseries \capatitulo\par}
    \end{center}
  \end{minipage}

  \vfill
  {\Large\bfseries
    Curitiba\\
    \capaano\par
  }

\end{center}
\newpage

\thispagestyle{empty}
\begin{center}
  \linespread{1.25}
  \sffamily

  {\Large \capaautor}

  \vspace{21em}

  \begin{minipage}{0.9\linewidth}
    \begin{center}
      {\LARGE\bfseries \capatitulo\par}
    \end{center}
  \end{minipage}

  \vspace{11em}

  \hfill
  \begin{minipage}{0.5\linewidth}
    \linespread{1.1}
    \large\rmfamily

    Monografia apresentada ao Programa de Especialização em Data Science
    e Big Data da Universidade Federal do Paraná como requisito
    parcial para a obtenção do grau de especialista.
    \newline

    Orientador: \capaorientador

  \end{minipage}

  \vfill
  {\Large\rmfamily
    Curitiba\\
    \capaano\par
  }

\end{center}
\newpage


%-----------------------------------------------------------------------

\twocolumn

\maketitle

% Para criar a chamada dos autores no rodapé.
\articlefootnotes

\begin{abstract}
  $resumo$
  
  \noindent\textbf{Palavras-chave}: $palavras_chave$.
\end{abstract}

\renewcommand{\abstractname}{Abstract}
\begin{abstract}
  \it
  $abstract$

  \noindent\textbf{Keywrods}: $keywords$.
\end{abstract}

%-----------------------------------------------------------------------

$body$

%-----------------------------------------------------------------------
% Referências bibliográficas. ------------------------------------------

\bibliographystyle{$biblio-style$}%
\bibliography{$references$}%

\end{document}

%-----------------------------------------------------------------------